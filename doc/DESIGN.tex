= wikitexer design document =

== Introduction ==

\notice{This is a work in progress!}

What would you give to have simple wiki markup mixed with LaTeX for
writing documents? That itch is being scratched by WikiTeXer.

\mediawiki\ is becoming something of a standard, so that markup makes
sense to support. Other markups may be supported by input handlers in
the future. For now we created output handlers for LaTeX and HTML.
This means, to support both outputs, rather then converting native
LaTeX or HTML to the other format we create special generators
(macro's) that can provide either. This avoids the complexity of
translation routines at the expense of creating generators through a
shared mechanism.

The point here is minimalism and simplicity. Parsers, in general, tend
to get very complicated mostly to adapt to 'strange' or unpredicatble
user behaviour. \wikitexer\ is not in that particular business. Nor
does it provide a complete solution to all formats. The basic premise
is to use LaTeX in a convenient fashion for writing a book (yes, that
is the purpose), or papers, and to allow clean HTML exports for review
and other purposes (for example exporting to Microsoft Word).

The input files are written as .tex files with Wiki markup extensions
and macros. \wikitexer\ is a command-line or library pre-parser which
outputs either pure LaTeX or pure HTML.

Much of the simplicity is achieved by ignoring stuff the parser does
not understand, but filling in what it understands. By default
\wikitexer\ uses stdin and stdout - which would allow it to be
embedded in a webserver. Another simplification is that wikitexer is 
paragraph based for macro expansion - much like TeX itself.

== Flow ==

The program flow is as follows:

# After parsing CLI options an output handler is chosen; currenly only HTML is fully supported
# Next a parser (WtParser) is created, with a hook to the output handler
# WikiTexer is created, which is the state engine
# The input file(s) are parser, line by line using wikitexer.addline
# addline keeps track of @par, and write\_paragraph will send a paragraph, when it is a unit
# write\_paragraph parses the @par through parse\_paragraph
# next calls into the write handler, in three steps:
## start\_par (writes <p>)
## write\_paragraph
## end\_par (writes </p>)
# parse\_paragraph calls all formatting rules
# html.write\_paragraph uses 'last\_env' to write all divs and transforms source code to HTML
#
== \mediawiki\ input handler ==

The \mediawiki\ input handler supports a minimalistic \mediawiki\ style
markup. Other markup (like LaTex) it will basically ignore, unless a
handler exists. So \wikitexer\ supports \mediawiki\ headers, emphasing text
and URLs initially. Initial support for:

LaTeX style commands

\begin{verbatim}
<nowiki>suppresses markup</nowiki>
\end{verbatim}

Headings like:

<nowiki>== Heading ==</nowiki>

Formatting:

''Italicized text''
'''Bold text'''
<u>underline new material</u>

External links:

[http://en.wikipedia.org Wikipedia]

Templates:

{{variable}} or {{macro}} or {{rubycall}}, or the more LaTeX-like
\variable, \macro and \rubycall (see below).


== LaTeX output handler ==

Markup is translated to native or 'pure' LaTex. Embedded LaTeX is left alone.


== HTML output handler ==

Markup is translated to HTML. Embedded LaTeX, in this handler, is
ignored. This is important. If HTML output is critical use \wikitexer\
macro's rather then LaTeX commands and register these with the HTML
handler.

== Variables ==

Variables defined with the newcommand are only visible to the LaTeX
output mechanism. Variables defined visible to wikitexer are available
to both with:

  \newvar{\linux}{\emph{Linux}}

which can be written as

  \newvar{\linux}{''Linux''}

== Macro's and templates ==

Macros are powerful (templates in \mediawiki\ speak). They prevent
typing things more than once by defining shared attributes in a single
location.  LaTeX is a macro language, but not always that nice to
program.  Embedding Ruby and Ruby macro's is a great feature to add to
LaTeX' native macros. When using the {{call}} syntax, or LaTex style
backslash \call the input handler will try to find a matching routine
and use that. If it misses a routine it will assume it can ignore it.

LaTex uses the following syntax variations:

	\hfill
  \hyphenpenalty=5000
	\caption{This is the caption.\label{fig:rawss}}
  \epsfig{file=graph.ps, angle=-90, width=\linewidth}

  \begin{figure} ... \end{figure}
  \begin{minipage}[t]{0.58\linewidth} ... \end{minipage} % with (default parameter [t]).

See also [http://en.wikibooks.org/wiki/LaTeX/Customizing_LaTeX newcommand and newenvironment].

It would be nice to have Ruby macro's 'override' LaTeX commands - so
they can share the same syntax. This way different output handlers can
choose to implement either in LaTeX or Ruby. It would be cool to do
that from within TeX - who dares - but a preparser will do the job for
most situations - just expand to the point it becomes pure LaTeX. Say
we have defined a macro named 'expand' this could be implemented as:

  \expand{'text'}

if a Ruby method 'expand' exists in this context it will be called
with parameter text. Otherwise it is not expanded. Likewise we can 
define environment methods:

  \begin{expand}[s]{pars}
	\end{expand}

which will call into expand_begin and expand_end, if they exist ('s'
is a default parameter).

An environment is maintained by the document to provide output that
runs across multiple paragraphs. Environments are stacked where the
begin and end marker defines output, but also the stack position
defines output behaviour. For example in a verbatim environment the
HTML generator may have to output end of lines in certain
implementations (though a simple <pre></pre> may suffice).

Adding macro's is easy - with LaTeX add the macro's as usual. With
Ruby add the Ruby module on the wikitexer command line.

A Ruby macro is defined in a separate file as:

<source lang="ruby">
  module WikiTexer
		# Expand string s
		def expand s
			'hello '+s+' expanded'
		end
	end
</source>

Finally there is the option to call directly into ruby with:

  \ruby{Time.now}                     % => Tue Oct 21 13:38:01 +0200 2008
	\ruby{'Test'.sub(/[Tt]/,'x')}       % => xest
	\ruby{`cat /etc/hostname`}          % you are joking, right?

obviously a very powerful feature that needs to be disabled on a public web server.

Allowing more complicated multi lines:

	\ruby{{
    result = 0
    (1..10) do { | i | result += i }
		result
	}}


=== Example: insert table from file ===
  
One example is, based on whether we are generating LaTeX or HTML we
want to insert different code in the output handler. The injected
source can be stored in separate files. E.g.

  \insertfile{table1}

where the LaTeX version would use table1.tex, using a standard Tex
\input command, and the HTML version table1.html using a Ruby method
which can do more, for example strip extraneous HTML tags. The
alternative is to also parse the file:

  \parsefile{table1}

which expands macros in that file.

=== Example: fetch hot info ===

Another example would be to fetch live data from a database or a
webserver to generate a document. For example:

  \sqlfetch{connection=\sqlconnection, query='select record from db'}

or more likely

  \userinfo(\username)

with a Ruby implementation for both LaTeX and HTML output generators.


== See also ==

Ruby Mediacloth. A \mediawiki\ to HTML converter in Ruby. Appears to
lacking in functionality - but what is there may be of interest.

Python wikimarkup on Google's code. A basic \mediawiki\ to HTML
converter which can be run from the command line. Does not support
LaTeX.

latex2html - HTML generator written in Perl and rather ancient. Took
the effort of trying to understand LaTeX and convert that to HTML. An
impossible job. Not flexible in terms of generation.
