= WikiTeXer design document =

== Introduction ==

What would I give to have simple wiki markup mixed with LaTeX for
writing documents? That itch is being scratched here.

MediaWiki is becoming something of a standard, so that markup makes
sense to support. Other markups may be supported by input handlers in
the future. We create output handlers for LaTeX and HTML. This means,
to support both outputs, rather then converting native LaTeX or HTML
to the other format we create special generators (macro's) that can
provide either. This avoids the complexity of translation routines at the
expense of creating generators through a shared mechanism.

The point here is minimalism and simplicity! Parser tend to get very 
complicated as they have to adapt to strange user behaviour. WikiTeXer
is not in that business. Nor does it provide a complete solution to
all formats. The basic premise is to use LaTeX in a convenient fashion
for writine a book (yes, that is the purpose) and allow HTML exports
for review and other purposes.


== MediaWiki input handler ==

The MediaWiki input handler will support a minimalistic MediaWiki style 
markup. Other markup (like LaTex) it will basically ignore. So it should
support MediaWiki headers, emphasing text and URLs initially.


== LaTeX output handler ==

Markup is translated to native LaTex. Embedded LaTeX is left alone.


== HTML output handler ==

Markup is translated to HTML. Embedded LaTeX is ignored. This is
important. If HTML output is critical use macro's rather then LaTeX
commands and register these with the HTML handler.


== Macro's ==

LaTeX is a macro language, but not always that nice to program.
Embedding Ruby and Ruby macro's is a great feature to add to LaTeX
native macros.
