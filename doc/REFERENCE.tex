
= REFERENCE =

== Introduction ==

\wikitexer\ comes with a number of built-in functions. Firstly the
supported \mediawiki\ markup. Next there are built-in functions that
replace \LaTeX\ commands. 

== \mediawiki\ markup ==

* Equal signs for headers
* Double single quotes ''emphasis''
* Triple single quotes '''italic'''
* Underscore __underscore__

== \LaTeX\ commands ==

The following functions are defined:

* var: displays a \var{variable}
* fn:  displays a \fn{filename}
* func or function displays \func{function name}
* data displays \data{data}
* name displays \name{name}
* code displays \code{code}

The following environments are defined:

* verbatim
* quote and quotation
* ruby, python, perl, c, make, shell

=== \% ====

Comments and \% are handled the \latex\ way by \wikitexer\

=== \input ===

As \wiktexer\ needs to parse included files it uses the
\insertfile{name} command. Maybe it should accept the standard
\input\ command too.

== Special commands ==
